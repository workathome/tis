\documentclass[11pt,letterpaper]{report}

\usepackage[letterpaper]{geometry}

\title{\Huge Parte-B}
\author{Work at Home Soft}
\usepackage[utf8]{inputenc}
\usepackage[spanish]{babel}
\usepackage{latexsym,amsmath,amssymb,amsthm}
\usepackage{graphicx}
\usepackage{pifont}
\usepackage[pdftex=true,colorlinks=true,plainpages=false]{hyperref}
\hypersetup{urlcolor=blue}
\hypersetup{linkcolor=black}
\hypersetup{citecolor=black}
\usepackage{lastpage}
\usepackage{url}
\usepackage{anysize}
\marginsize{3cm}{3cm}{1.2cm}{2.3cm}
\usepackage{fancyhdr}
\usepackage{anyfontsize}
\usepackage{tocbibind}
\usepackage{eso-pic}
\usepackage{mathptmx}
\usepackage{draftwatermark}
\usepackage{multirow}
\usepackage{pdfpages}
\usepackage{array}
\usepackage{booktabs}% http://ctan.org/pkg/booktabs
\newcommand{\tabitem}{~~\llap{\textbullet}~~}
\usepackage{multirow}
\usepackage{longtable}
\usepackage{lscape}
\usepackage{pdflscape}
\usepackage{fancyref}
%\usepackage[usenames,dvipsnames]{xcolor}
%\usepackage{lipsum}
%-------------------------
\setlength{\parindent}{0in}
\pagestyle{fancy}
\lhead{}
\chead{{\Huge \tt Work at Home Soft S.R.L.}\\~\\~ }%\\{\tt \huge ~}\\{\tt \Large -~}}
\rhead{}
\lfoot{}
\cfoot{{\small \tt Av. Maria del Carmen Rodriguez s/n, Zona Pacata Baja Cochabamba.\\ Teléfono: +591-70797024}\\{\Large \tt \url{http://github.com/workathome}}}
\rfoot{\thepage/\pageref{LastPage}}
\renewcommand{\headrule}{{\color{black}\hrule width\headwidth height\headrulewidth \vskip-\headrulewidth}}
\renewcommand{\footrule}{{\color{black}%
		\vskip-\footruleskip\vskip-\footrulewidth
		\hrule width\headwidth height\footrulewidth\vskip\footruleskip}}
\renewcommand{\headrulewidth}{3pt}
\renewcommand{\footrulewidth}{3pt}
\footskip = 32pt
\headheight = 56pt
\headsep = 8pt
%--------------------------


\newcommand\BackgroundPic{
	\put(50,707){
		\parbox[b][\paperheight]{\paperwidth}{%
			%\vfill
			%\centering
			\includegraphics[scale=0.83]{../cartas/img/logo.png}%
			%\vfill
		}}}
		
		%---------------------------
		\SetWatermarkAngle{0}
		\SetWatermarkLightness{0.1}
		%\SetWatermarkFontSize{1cm}
		\SetWatermarkScale{3}
		\SetWatermarkText{\includegraphics[scale=0.33]{../cartas/img/marca_agua.png}}
		%--------------------------------------		--------------------
		
\begin{document}
	
	\tableofcontents
	
	\section{Historia de usuario 1}
	\begin{center}	
	\begin{tabular}{|p{12cm}|}
		\hline
		\begin{tabular}{c|c}
			\textbf{HISTORIA DE USUARIO} & \textbf{Número: 1} \\
		\end{tabular} \\ \hline
		\textbf{Usuario} (Yo como): Administrador \\ \hline
		\textbf{Nombre de la historia} (Requiero): Administración de cuentas. \\ \hline
		\textbf{Descripción} (Para): Crear cuentas de consultores y dar de baja las cuentas que sean necesarias. \\ \hline
		\textbf{Validación} (Criterios de aceptación): \\
		\begin{minipage}{12cm}
			\begin{itemize}
				\item El administrador solamente podrá registrar cuentas de consultores.
				\item El administrador solamente podrá dar de baja cuentas de consultores y de grupo empresas.
				\item Cada consultor deberá tener acceso a su cuenta mediante un nombre único y una contraseña.
				\item Ningún usuario distinto al dueño de la cuenta deberá poder ingresar a la misma.
			\end{itemize}
		\end{minipage} \\ \hline
				\begin{tabular}{p{6cm}|c}
					\textbf{Prioridad en el negocio: } & \textbf{Puntos de esfuerzo: 3 galletas} \\
				\end{tabular} \\ \hline
	\end{tabular}
	\end{center}
	
	\section{Historia de usuario 2}
	
	\begin{center}	
	\begin{tabular}{|p{12cm}|}
		\hline
		\begin{tabular}{c|c}
			\textbf{HISTORIA DE USUARIO} & \textbf{Número: 2} \\
		\end{tabular} \\ \hline
		\textbf{Usuario} (Yo como): Grupo – Empresa \\ \hline
		\textbf{Nombre de la historia} (Requiero): Ingresar al sistema de forma segura. \\ \hline
		\textbf{Descripción} (Para): Ingresar como un usuario que tiene permisos para realizar las consultas y deberes específicos como Grupo – Empresa. \\ \hline
		\textbf{Validación} (Criterios de aceptación): \\
		\begin{minipage}{12cm}
			\begin{itemize}
				\item Las Grupo - Empresas deberán poder registrarse con una cuenta única, que represente a toda la grupo-empresa.
				\item Cada Grupo – Empresa deberá poder acceder a su cuenta mediante un nombre único y una contraseña.
				\item Ningún usuario distinto al dueño de la cuenta deberá poder ingresar a la misma.
			\end{itemize}
		\end{minipage} \\ \hline
		\begin{tabular}{p{6cm}|c}
			\textbf{Prioridad en el negocio: } & \textbf{Puntos de esfuerzo: 3 galletas } \\
		\end{tabular} \\ \hline
	\end{tabular}
	\end{center}
	
	\section{Historia de usuario 3}
	
	\begin{center}	
	\begin{tabular}{|p{12cm}|}
		\hline
		\begin{tabular}{c|c}
			\textbf{HISTORIA DE USUARIO} & \textbf{Número: 3} \\
		\end{tabular} \\ \hline
		\textbf{Usuario} (Yo como): Consultor \\ \hline
		\textbf{Nombre de la historia} (Requiero): Publicar documentos públicos en el Sistema. \\ \hline
		\textbf{Descripción} (Para): Las grupo-empresas pueden descargar el documento publicado. \\ \hline
		\textbf{Validación} (Criterios de aceptación): \\
		\begin{minipage}{12cm}
			\begin{itemize}
				\item El documento debe poder publicarse por algún consultor en formato PDF.
				\item Cada Grupo – Empresa deberá poder acceder a su cuenta mediante un nombre único y una contraseña.
				\item El documento solamente deberá poder descargarse.
			\end{itemize}
		\end{minipage} \\ \hline
		\begin{tabular}{p{6cm}|c}
			\textbf{Prioridad en el negocio: } & \textbf{Puntos de esfuerzo: 3 galletas } \\
		\end{tabular} \\ \hline
	\end{tabular}
	\end{center}
	
	\section{Historia de usuario 4}
	
	\begin{center}	
		\begin{tabular}{|p{12cm}|}
			\hline
			\begin{tabular}{c|c}
				\textbf{HISTORIA DE USUARIO} & \textbf{Número: 4} \\
			\end{tabular} \\ \hline
			\textbf{Usuario} (Yo como): grupo – empresa \\ \hline
			\textbf{Nombre de la historia} (Requiero): Descargar documentos públicos subidos por los consultores. \\ \hline
			\textbf{Descripción} (Para): Poder visualizar el documento en formato PDF. \\ \hline
			\textbf{Validación} (Criterios de aceptación): \\
			\begin{minipage}{12cm}
				\begin{itemize}
					\item El documento solamente deberá poder descargarse.
				\end{itemize}
			\end{minipage} \\ \hline
			\begin{tabular}{p{6cm}|c}
				\textbf{Prioridad en el negocio: } & \textbf{Puntos de esfuerzo: 3 galletas} \\
			\end{tabular} \\ \hline
		\end{tabular}
	\end{center}
	
	\section{Historia de usuario 5}	
	
		\begin{center}	
			\begin{tabular}{|p{12cm}|}
				\hline
				\begin{tabular}{c|c}
					\textbf{HISTORIA DE USUARIO} & \textbf{Número: 5} \\
				\end{tabular} \\ \hline
				\textbf{Usuario} (Yo como): Consultor \\ \hline
				\textbf{Nombre de la historia} (Requiero): Establecer fechas o hitos de la propuesta(Actividades). \\ \hline
				\textbf{Descripción} (Para): Definir fechas para la entrega de propuestas, ordenes de cambio y firma del contrato. \\ \hline
				\textbf{Validación} (Criterios de aceptación): \\
				\begin{minipage}{12cm}
					\begin{itemize}
						\item Estas fechas deben ser visibles para todas las grupo - empresas y consultores.
						\item Estas fechas serán únicamente modificadas por los consultores..
					\end{itemize}
				\end{minipage} \\ \hline
				\begin{tabular}{p{6cm}|c}
					\textbf{Prioridad en el negocio: } & \textbf{Puntos de esfuerzo: 7 galletas} \\
				\end{tabular} \\ \hline
			\end{tabular}
		\end{center}
	
	\section{Historia de usuario 6}
	\begin{center}	
		\begin{tabular}{|p{12cm}|}
			\hline
			\begin{tabular}{c|c}
				\textbf{HISTORIA DE USUARIO} & \textbf{Número: 6} \\
			\end{tabular} \\ \hline
			\textbf{Usuario} (Yo como): Grupo – Empresa \\ \hline
			\textbf{Nombre de la historia} (Requiero): Registrar los datos más importantes de los socios de la grupo – empresa. \\ \hline
			\textbf{Descripción} (Para): El consultor podrá acceder a estos datos de ser necesario. \\ \hline
			\textbf{Validación} (Criterios de aceptación): \\
			\begin{minipage}{12cm}
				\begin{itemize}
					\item El registro consta de los siguientes datos ( nombre completo , dirección, teléfono)
					\item Debe diferenciarse el representante legal del resto de los socios.
					\item El consultor podrá solamente ver la información de los socios de las grupo – empresas.
				\end{itemize}
			\end{minipage} \\ \hline
			\begin{tabular}{p{6cm}|c}
				\textbf{Prioridad en el negocio: } & \textbf{Puntos de esfuerzo: 4 galletas} \\
			\end{tabular} \\ \hline
		\end{tabular}
	\end{center}
	
	\section{Historia de usuario 7}
	
	\begin{center}	
		\begin{tabular}{|p{12cm}|}
			\hline
			\begin{tabular}{c|c}
				\textbf{HISTORIA DE USUARIO} & \textbf{Número: 7} \\
			\end{tabular} \\ \hline
			\textbf{Usuario} (Yo como): Grupo – Empresa \\ \hline
			\textbf{Nombre de la historia} (Requiero): Fijar un día de reunión o seguimiento semanal. \\ \hline
			\textbf{Descripción} (Para): Tener fijo un día de reuniones en los cuales se registrara el avance del proyecto. \\ \hline
			\textbf{Validación} (Criterios de aceptación): \\
			\begin{minipage}{12cm}
				\begin{itemize}
					\item Este día seleccionado deberá mostrarse cada semana en el sistema de seguimiento del proyecto.
				\end{itemize}
			\end{minipage} \\ \hline
			\begin{tabular}{p{6cm}|c}
				\textbf{Prioridad en el negocio: } & \textbf{Puntos de esfuerzo: 4 galletas} \\
			\end{tabular} \\ \hline
		\end{tabular}
	\end{center}
	
	\section{Historia de usuario 8}
	
	\begin{center}	
		\begin{tabular}{|p{12cm}|}
			\hline
			\begin{tabular}{c|c}
				\textbf{HISTORIA DE USUARIO} & \textbf{Número: 8} \\
			\end{tabular} \\ \hline
			\textbf{Usuario} (Yo como): Consultor \\ \hline
			\textbf{Nombre de la historia} (Requiero): Generar un contrato \\ \hline
			\textbf{Descripción} (Para): Generar automáticamente un contrato con un formato establecido y con los respectivos datos de las grupo-empresas.\\ \hline
			\textbf{Validación} (Criterios de aceptación): \\
			\begin{minipage}{12cm}
				\begin{itemize}
					\item El contrato deberá mostrarse en un formato PDF.
					\item El contrato deberá generarse con los siguientes datos requeridos (nombre largo y corto de la empresa, código de referencia a documento CPTIS, nombres de los consultores y nombre representante legal).
				\end{itemize}
			\end{minipage} \\ \hline
			\begin{tabular}{p{6cm}|c}
				\textbf{Prioridad en el negocio: } & \textbf{Puntos de esfuerzo: 10 galletas} \\
			\end{tabular} \\ \hline
		\end{tabular}
	\end{center}
	
	\section{Historia de usuario 9}
	
	\begin{center}	
		\begin{tabular}{|p{12cm}|}
			\hline
			\begin{tabular}{c|c}
				\textbf{HISTORIA DE USUARIO} & \textbf{Número: 9} \\
			\end{tabular} \\ \hline
			\textbf{Usuario} (Yo como): Grupo – Empresa \\ \hline
			\textbf{Nombre de la historia} (Requiero): Subir parte A de la propuesta. \\ \hline
			\textbf{Descripción} (Para): Subir la parte A de la propuesta al sistema. \\ \hline
			\textbf{Validación} (Criterios de aceptación): \\
			\begin{minipage}{12cm}
				\begin{itemize}
					\item El formato de la parte A de la propuesta solo se podrá subir en formato PDF.
					\item Debe registrarse la hora y fecha de la subida de la parte A.
					\item El documento solo lo podrán descargar la grupo - empresa a la que pertenece y el consultor.
				\end{itemize}
			\end{minipage} \\ \hline
			\begin{tabular}{p{6cm}|c}
				\textbf{Prioridad en el negocio: } & \textbf{Puntos de esfuerzo: 4 galletas} \\
			\end{tabular} \\ \hline
		\end{tabular}
	\end{center}
	
	\section{Historia de usuario 10}
	
	\begin{center}	
		\begin{tabular}{|p{12cm}|}
			\hline
			\begin{tabular}{c|c}
				\textbf{HISTORIA DE USUARIO} & \textbf{Número: 10} \\
			\end{tabular} \\ \hline
			\textbf{Usuario} (Yo como): Grupo – Empresa \\ \hline
			\textbf{Nombre de la historia} (Requiero): Subir parte B de la propuesta.  \\ \hline
			\textbf{Descripción} (Para): Subir la parte B de la propuesta al
			sistema. \\ \hline
			\textbf{Validación} (Criterios de aceptación): \\
			\begin{minipage}{12cm}
				\begin{itemize}
					\item El formato de la parte B de la propuesta solo se podrá subir en formato PDF.
					\item Debe registrarse la hora y fecha de la subida de la parte B.
					\item El documento solo lo podrán descargar la grupo-empresa a la que pertenece y el consultor.
				\end{itemize}
			\end{minipage} \\ \hline
			\begin{tabular}{p{6cm}|c}
				\textbf{Prioridad en el negocio: } & \textbf{Puntos de esfuerzo: 4 galletas} \\
			\end{tabular} \\ \hline
		\end{tabular}
	\end{center}
	
	\section{Historia de usuario 11}
	
	\begin{center}	
		\begin{tabular}{|p{12cm}|}
			\hline
			\begin{tabular}{c|c}
				\textbf{HISTORIA DE USUARIO} & \textbf{Número: 11} \\
			\end{tabular} \\ \hline
			\textbf{Usuario} (Yo como): Consultor \\ \hline
			\textbf{Nombre de la historia} (Requiero): Lista de las propuestas subidas al sistema. \\ \hline
			\textbf{Descripción} (Para): Poder descargar las propuestas (parte A, parte B) y diferenciarlas de las que llegaron a destiempo. \\ \hline
			\textbf{Validación} (Criterios de aceptación): \\
			\begin{minipage}{12cm}
				\begin{itemize}
					\item Diferenciar las propuestas por fecha de subida (las propuestas que llegaron a destiempo, deben diferenciarse de las demás).
				\end{itemize}
			\end{minipage} \\ \hline
			\begin{tabular}{p{6cm}|c}
				\textbf{Prioridad en el negocio: } & \textbf{Puntos de esfuerzo: 7 galletas} \\
			\end{tabular} \\ \hline
		\end{tabular}
	\end{center}
	
	\section{Historia de usuario 12}
	
	\begin{center}	
		\begin{tabular}{|p{12cm}|}
			\hline
			\begin{tabular}{c|c}
				\textbf{HISTORIA DE USUARIO} & \textbf{Número: 12} \\
			\end{tabular} \\ \hline
			\textbf{Usuario} (Yo como): Grupo – Empresa \\ \hline
			\textbf{Nombre de la historia} (Requiero): Registrar un avance semanal. \\ \hline
			\textbf{Descripción} (Para): Poder demostrar que la empresa está cumpliendo con el plan especificado. \\ \hline
			\textbf{Validación} (Criterios de aceptación): \\
			\begin{minipage}{12cm}
				\begin{itemize}
					\item Los registros deberán ser realizados el día acordado según calendario.
					\item Los avances registrados, los podrán visualizar tanto las grupo empresas como los consultores.
					\item El registro de los avances serán de forma literal.
					\item Los avances registrados no se podrán modificar ni eliminar por ningún consultor o grupo empresa.
				\end{itemize}
			\end{minipage} \\ \hline
			\begin{tabular}{p{6cm}|c}
				\textbf{Prioridad en el negocio: } & \textbf{Puntos de esfuerzo: 9 galletas} \\
			\end{tabular} \\ \hline
		\end{tabular}
	\end{center}
	
	\section{Historia de usuario 13}
	
	\begin{center}	
		\begin{tabular}{|p{12cm}|}
			\hline
			\begin{tabular}{c|c}
				\textbf{HISTORIA DE USUARIO} & \textbf{Número: 13} \\
			\end{tabular} \\ \hline
			\textbf{Usuario} (Yo como): Consultor \textbf{OBSERVADO}\\ \hline
			\textbf{Nombre de la historia} (Requiero): Visualizar la intención de avance de cada grupo-empresa y registrar el avance realizado \\ \hline
			\textbf{Descripción} (Para): Verificar que cada grupo-empresa está cumpliendo con el plan propuesto. \\ \hline
			\textbf{Validación} (Criterios de aceptación): \\
			\begin{minipage}{12cm}
				\begin{itemize}
					\item El consultor podrá visualizar la intención de avance de cada grupo empresa.
					\item El consultor podrá registrar un detalle acerca el avance de cada grupo empresa.
					\item El consultor podrá registrar observaciones acerca el avance de cada grupo empresa.
					\item El registro de observaciones solo debe ser visualizado por el mismo consultor.
				\end{itemize}
			\end{minipage} \\ \hline
			\begin{tabular}{p{6cm}|c}
				\textbf{Prioridad en el negocio: } & \textbf{Puntos de esfuerzo: 8 galletas} \\
			\end{tabular} \\ \hline
		\end{tabular}
	\end{center}
	
	\section{Historia de usuario 14}
	
	\begin{center}	
		\begin{tabular}{|p{12cm}|}
			\hline
			\begin{tabular}{c|c}
				\textbf{HISTORIA DE USUARIO} & \textbf{Número: 14} \\
			\end{tabular} \\ \hline
			\textbf{Usuario} (Yo como): Consultor \\ \hline
			\textbf{Nombre de la historia} (Requiero): una forma de aclarar a las grupo empresas acerca de la convocatoria y el pliego de especificaciones \\ \hline
			\textbf{Descripción} (Para): poder aclarar cualquier duda de las grupo empresas acerca de la convocatoria y el pliego de especificaciones \\ \hline
			\textbf{Validación} (Criterios de aceptación): \\
			\begin{minipage}{12cm}
				\begin{itemize}
					\item Proponer y responder dudas a través de un foro.
					\item Cualquier usuario puede acceder al foro.
				\end{itemize}
			\end{minipage} \\ \hline
			\begin{tabular}{p{6cm}|c}
				\textbf{Prioridad en el negocio: } & \textbf{Puntos de esfuerzo: 8 galletas} \\
			\end{tabular} \\ \hline
		\end{tabular}
	\end{center}
	
	\section{Historia de usuario 15}
	
	\begin{center}	
		\begin{tabular}{|p{12cm}|}
			\hline
			\begin{tabular}{c|c}
				\textbf{HISTORIA DE USUARIO} & \textbf{Número: 15} \\
			\end{tabular} \\ \hline
			\textbf{Usuario} (Yo como): Consultor \\ \hline
			\textbf{Nombre de la historia} (Requiero): Visualizar bitácora de actividades\\ \hline
			\textbf{Descripción} (Para): Poder ver todas las actividades de las grupo empresas en el sistema \\ \hline
			\textbf{Validación} (Criterios de aceptación): \\
			\begin{minipage}{12cm}
				\begin{itemize}
					\item Todas las actividades de subida de archivos que realicen las grupo empresas deben registrarse en la bitácora.
					\item La bitácora de actividades podrá ser visualizada por las grupo-empresas y el consultor respectivo.
					\item Todo registro de actividad contara con hora y fecha.
				\end{itemize}
			\end{minipage} \\ \hline
			\begin{tabular}{p{6cm}|c}
				\textbf{Prioridad en el negocio: } & \textbf{Puntos de esfuerzo: 7 galletas} \\
			\end{tabular} \\ \hline
		\end{tabular}
	\end{center}
	
	\section{Historia de usuario 16}
	
	\begin{center}	
		\begin{tabular}{|p{12cm}|}
			\hline
			\begin{tabular}{c|c}
				\textbf{HISTORIA DE USUARIO} & \textbf{Número: 16} \\
			\end{tabular} \\ \hline
			\textbf{Usuario} (Yo como): Consultor \\ \hline
			\textbf{Nombre de la historia} (Requiero): Crear un proyecto \\ \hline
			\textbf{Descripción} (Para): Poder registrar un código del proyecto. \\ \hline
			\textbf{Validación} (Criterios de aceptación): \\
			\begin{minipage}{12cm}
				\begin{itemize}
					\item Las grupo-empresas puedan registrarse según el proyecto vigente.
					\item Solo puede existir un proyecto vigente.
					\item El proyecto debe poderse dar de baja.
				\end{itemize}
			\end{minipage} \\ \hline
			\begin{tabular}{p{6cm}|c}
				\textbf{Prioridad en el negocio: } & \textbf{Puntos de esfuerzo: 5 galletas} \\
			\end{tabular} \\ \hline
		\end{tabular}
	\end{center}
	
	\section{Historia de usuario 17}
	
	\begin{center}	
		\begin{tabular}{|p{12cm}|}
			\hline
			\begin{tabular}{c|c}
				\textbf{HISTORIA DE USUARIO} & \textbf{Número: 17} \\
			\end{tabular} \\ \hline
			\textbf{Usuario} (Yo como): Grupo-empresa \\ \hline
			\textbf{Nombre de la historia} (Requiero): Registrar un plan de pagos \\ \hline
			\textbf{Descripción} (Para): poder tener un calendario de seguimiento. \\ \hline
			\textbf{Validación} (Criterios de aceptación): \\
			\begin{minipage}{12cm}
				\begin{itemize}
					\item Los hitos del plan no pueden ser modificados.
					\item Las fechas de pago deben ser sincronizadas con un calendario.
					\item La suma total de los montos por cada hito debe ser el total a pagar por el proyecto.
				\end{itemize}
			\end{minipage} \\ \hline
			\begin{tabular}{p{6cm}|c}
				\textbf{Prioridad en el negocio: } & \textbf{Puntos de esfuerzo: 9 galletas} \\
			\end{tabular} \\ \hline
		\end{tabular}
	\end{center}
	
	\section{Historia de usuario 18}
	
	\begin{center}	
		\begin{tabular}{|p{12cm}|}
			\hline
			\begin{tabular}{c|c}
				\textbf{HISTORIA DE USUARIO} & \textbf{Número: 18} \\
			\end{tabular} \\ \hline
			\textbf{Usuario} (Yo como): Consultor \\ \hline
			\textbf{Nombre de la historia} (Requiero): realizar una evaluación final \\ \hline
			\textbf{Descripción} (Para): Determinar la nota final de cada grupo-empresa \\ \hline
			\textbf{Validación} (Criterios de aceptación): \\
			\begin{minipage}{12cm}
				\begin{itemize}
					\item Los criterios de evaluación pueden ser ilimitados
					\item El tipo de calificación de cada criterio puede ser diferente para cada uno de ellos.
					\item Cada criterio puede tener un porcentaje de calificación independiente.
				\end{itemize}
			\end{minipage} \\ \hline
			\begin{tabular}{p{6cm}|c}
				\textbf{Prioridad en el negocio: } & \textbf{Puntos de esfuerzo: 6 galletas} \\
			\end{tabular} \\ \hline
		\end{tabular}
	\end{center}
	
	\section{Historia de usuario 19}	
	
	\begin{center}	
		\begin{tabular}{|p{12cm}|}
			\hline
			\begin{tabular}{c|c}
				\textbf{HISTORIA DE USUARIO} & \textbf{Número: 19} \\
			\end{tabular} \\ \hline
			\textbf{Usuario} (Yo como): Consultor \\ \hline
			\textbf{Nombre de la historia} (Requiero): registro de hitos aprobados \\ \hline
			\textbf{Descripción} (Para): Tener un seguimiento de los hitos aprobados por el consultor a la grupo-empresa. \\ \hline
			\textbf{Validación} (Criterios de aceptación): \\
			\begin{minipage}{12cm}
				\begin{itemize}
					\item Los valores de los pagos establecidos por la grupo empresa no pueden ser modificados por ningún consultor o grupo empresa una vez establecido.
					\item La fecha de pago debe hacer referencia a un hito pagable registrado por la grupo empresa.
					\item Se registrara el número de factura correspondiente a cada pago.
					\item El valor de un pago nunca puede ser negativo.
				\end{itemize}
			\end{minipage} \\ \hline
			\begin{tabular}{p{6cm}|c}
				\textbf{Prioridad en el negocio: } & \textbf{Puntos de esfuerzo: 6 galletas} \\
			\end{tabular} \\ \hline
		\end{tabular}
	\end{center}
	
\end{document}