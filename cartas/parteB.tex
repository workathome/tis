\documentclass[11pt,letterpaper]{report}
\usepackage[utf8]{inputenc}
\usepackage[spanish]{babel}
\usepackage{latexsym,amsmath,amssymb,amsthm}
\usepackage{graphicx}
\usepackage{pifont}
\usepackage[pdftex=true,colorlinks=true,plainpages=false]{hyperref}
\hypersetup{urlcolor=blue}
\hypersetup{linkcolor=black}
\hypersetup{citecolor=black}
\usepackage{lastpage}
\usepackage{url}
\usepackage{anysize}
\marginsize{3cm}{3cm}{1.2cm}{2.3cm}
\usepackage{fancyhdr}
\usepackage{anyfontsize}
\usepackage{tocbibind}
\usepackage{eso-pic}
\usepackage{mathptmx}
\usepackage{draftwatermark}
\usepackage{multirow}
\usepackage{pdfpages}
\usepackage{array}
\usepackage{multirow}
%\usepackage[usenames,dvipsnames]{xcolor}
%\usepackage{lipsum}
%-------------------------
\setlength{\parindent}{0in}
\pagestyle{fancy}
\lhead{}
\chead{{\Huge \tt Work at Home Soft S.R.L.}\\~\\~ }%\\{\tt \huge ~}\\{\tt \Large -~}}
\rhead{}
\lfoot{}
\cfoot{{\small \tt Av. Maria del Carmen Rodriguez s/n, Zona Pacata Baja Cochabamba.\\ Teléfono: +591-70797024}\\{\Large \tt \url{http://github.com/workathome}}}
\rfoot{\thepage/\pageref{LastPage}}
\renewcommand{\headrule}{{\color{black}\hrule width\headwidth height\headrulewidth \vskip-\headrulewidth}}
\renewcommand{\footrule}{{\color{black}%
\vskip-\footruleskip\vskip-\footrulewidth
\hrule width\headwidth height\footrulewidth\vskip\footruleskip}}
\renewcommand{\headrulewidth}{3pt}
\renewcommand{\footrulewidth}{3pt}
\footskip = 32pt
\headheight = 56pt
\headsep = 8pt
%--------------------------


\newcommand\BackgroundPic{
\put(50,707){
\parbox[b][\paperheight]{\paperwidth}{%
%\vfill
%\centering
\includegraphics[scale=0.83]{img/logo.png}%
%\vfill
}}}

%---------------------------
\SetWatermarkAngle{0}
\SetWatermarkLightness{0.1}
%\SetWatermarkFontSize{1cm}
\SetWatermarkScale{3}
\SetWatermarkText{\includegraphics[scale=0.33]{img/marca_agua.png}}
%--------------------------------------

\begin{document}
\AddToShipoutPicture{\BackgroundPic}
%	\tableofcontents
~\\
~\\

\tableofcontents

\chapter{PROPUESTA DE SERVICIOS}
\section{Descripción del sistema a desarrollar}
La empresa TIS es una empresa que se dedica
al asesoramiento de grupos de desarrollo de software. Este trabajo consiste en acompañar el trabajo
de un grupo-empresa en el desarrollo de un software. La observación del trabajo de acompañamiento
permite a los asesores indicar los puntos débiles de un grupo empresa para que la calidad de su
proceso de desarrollo pueda mejorar.
La empresa TIS ha decidido desarrollar un sistema computacional para automatizar sus procesos.
Actualmente la forma en que suceden los procesos es la siguiente:
\begin{itemize}
	\item Publicación de la convocatoria pública.
	\item Publicación de lista de empresas inscritas en fundempresa de TIS.
	\item Publicación del pliego de especificaciones.
	\item Aclaraciones a la convocatoria y al pliego de especificaciones.
	\item Recepción de las propuestas e inscripción de las grupo empresas, cosiderando sus especificidades.
	\item Llevar una historia de las actividades en el desarrollo de la convocatoria.
	\item Permitir la emisión de  órdenes de cambio de acuerdo a convocatoria.
	\item Permitir la emisión de contratos.
	\item Permitir el registro de los avances semanales de las grupo empresas.
	\item Seguimiento de las grupo empresas, acorde a los que las GE definan como marco de trabajo. Una guia general puede ser tomando un modelo general de seguimiento de proyectos: Actividad, tarea, indicador,fecha de inicio, fecha fin, responsable, producto, observaciones.
\end{itemize}

\section{Objetivo General}

Desarrollar un sistema computacional que permita llevar adelante
los procesos administrativos y de gestión de proyectos de la empresa TIS.

\section{Objetivos especificos}

\section{Modalidad del proyecto}
Work at Home Soft presenta su propuesta de servicios en la modalidad de desarrollo de un producto de software, tal como exige el pliego de especificaciones PETIS-1707-2014.
\subsection{Proceso de desarrollo}
Para el desarrollo del sistema de Apoyo TIS se aplicara una metodologia de desarrollo agil é hibrida "Scrum + Kanban".\\
Las fases de desarrollo del proyecto son:\\
\begin{tabular}{ |c|p{4cm}|c|c|r|r| }
	\hline
	\multicolumn{6}{ |c| }{Planificación} \\
	\hline
	\textbf{Fases} & \textbf{Etapa} & \textbf{Fecha Inicio} & \textbf{Fecha Fin} & \textbf{Dias} & \textbf{Costo Bs.-}\\ \hline
	1 & Firma Contrato & 2014-09-16 & 2014-09-19 & 1& 1234Bs. \\ \hline
	2 & Product Backlog & 2014-09-22 & 2014-09-26 & 5& 1234Bs. \\ \hline
	3 & Sprint 1 & 2014-09-29 & 2014-10-10 & 10& 1234Bs. \\ \hline
	4 & Sprint 2 & 2014-10-13 & 2014-10-24 & 10& 1234Bs. \\ \hline
	5 & Sprint 3 & 2014-10-27 & 2014-11-07 & 10& 1234Bs. \\ \hline
	6 & Sprint 4 & 2014-11-10 & 2014-11-21 & 10& 1234Bs. \\ \hline
	7 & Capacitación & 2014-11-24 & 2014-11-26 & 3& 1234Bs. \\ \hline
	7 & Transferencía de tecnologia, implantacion y puesta en marcha & 2014-11-27 & 2014-12-01 & 3& 1234Bs. \\ \hline
	\multicolumn{4}{|c|}{Total dias hábiles laborales} & 52 & 1222Bs. \\
	\hline
\end{tabular}
\section{Generalidades del sistema}

\subsection{Gestión de Base de Datos}
TO-DO
\subsection{Gestión de Información}
TO-DO

\section{Software para el desarrollo}
\subsection{Licencias de software}
Work at Home Soft en cumplimiento con el pliego de especificaciones asegura que la tecnologia y herramientas a utilizar  en el desarrollo del sistema tienen licencia libre, a continuacion se detalla:\\
\\
\begin{tabular}{ lccc }
	\multicolumn{4}{c}{Tecnologia de desarrollo} \\
	\hline
	\textbf{Descripción} & \textbf{Tecnologia} & \textbf{Licencia} & \textbf{Referecia} \\ \hline
	Lenguaje de Programación & PHP 5 & PHP License & \url{http://php.net/copyright.php}\\
	Framework Backend & Laravel & MIT License & \url{http://opensource.org/licenses/MIT} \\  
	Gestor de Base de Datos & Mariadb & GPL License & \url{http://www.gnu.org/copyleft/gpl.html} \\
	Diseño de Base de Datos & Mysql-WorkBench & GPL License & \url{http://www.gnu.org/copyleft/gpl.html}\\
	Control de Versiones & Git & GPL License & \url{http://www.gnu.org/copyleft/gpl.html}\\
	Entorno de desarrollo & Vim & GPL-compatible & \url{http://www.gnu.org/copyleft/gpl.html}\\
	
	
\end{tabular}
\\
\\Ademas de que el sistema desarrollado en la primera fase ya cumple con este requerimiento.
\section{Metodología del desarrollo}
Se usara una metodologia de desarrollo hibrida, una mezcla de Scrum y Kanban.\\
A continuacion se detalla la adopción de esta.
\subsubsection{Scrum:} 
Se usara lo siguiente:
\begin{itemize}
	\item Reuniones establecidas en scrum(Daily Meeting, Sprint Planning Meeting, Sprint Review Meeting, Sprint Retrospective).
	\item Roles de scrum(Product Owner, ScrumMaster, Team).
	\item Documentos de scrum(Product backlog, Sprint backlog, Burn down chart).
\end{itemize}
\subsubsection{Kanban:} 
\begin{itemize}
	\item Tablero Kanban.\\
	Un tablero Kanban, se divide en columnas las cuales representan un proceso de trabajo. En nuestro caso las columnas se dividen en Cola de Espera, Analisis, Desarrollo, Pruebas, Finalizado.
	\item Uso correcto de WIP( Work in Progress) Trabajo en progreso.\\
	Consisten en acordar anticipadamente, la cantidad de ítems que pueden abordarse por cada proceso (es decir, por columnas del tablero).\\
	El principal objetivo de establecer estos límites, es el de detectar cuellos de botella. 
	\item Optimizacion del Flujo de trabajo.\\
	El objetivo una la producción estable, continua y previsible. Midiendo el tiempo que el ciclo completo de ejecución del proyecto demanda, se obtiene el CycleTime(tiempo de ciclo).
	\begin{equation}
	Throughput(rendimiento) = CycleTime/WIP
	\end{equation}
	Con estos valores, la optimización del flujo de trabajo consistirá en la búsqueda de:
	\begin{enumerate}
		\item Minimizar el CycleTime(tiempo de ciclo).
		\item Maximizar el Throughput(redimeinto).
		\item Lograr una variabilidad mínima entre CycleTime y Throughput
	\end{enumerate}
\end{itemize}

\chapter{PLANIFICACIÓN}
\section{Planificacion global del proyecto}
A continuacion se presenta la planificación global de actividades acorde con nuestro proceso de desarrollo.
\begin{center}
\begin{tabular}{|c|p{4.5cm}|c|p{6cm}|}
	\hline
	\multicolumn{4}{|c|}{Planificación global} \\
	\hline
	\ Numero & Actividad & Duracion(Dias) & Hitos \\ \hline
	A1 & Firma del contrato & 1 & Contrato firmado. \\ \hline
	A2 & Product Backlog & 5 & Documento del Product Backlog. \\ \hline
	A3 & Sprint I & 10 & Primera versión del sistema. \\ \hline
	A4 & Sprint II & 10 & Segunda versión del sistema. \\ \hline
	A5 & Sprint III & 10 & Tercera versión del sistema. \\ \hline
	A6 & Sprint IV & 10 & Versión final del sistema. \\ \hline
	A7 & Capacitación & 3 &  Capacitacion a los usuarios finales.\\ \hline
	A8 & Transferencía de tecnologia, implantacion y puesta en marcha. & 3 & Implantacion de sistema, Codigo fuente, manual técnico  de usuario y de instalación. \\  \hline
\end{tabular}
\end{center}
\section{Planificacion especifica del proyecto}
\begin{tabular}{|c|p{4cm}|c|c|l|p{3cm}|}
	\hline
	\multicolumn{6}{|c|}{Planificacion especifica del proyecto} \\ \hline
	Actividad & Plan & Fecha Inicio & Fecha Fin & Responsables & Detalle \\ \hline
	Product & Gestion de Historias  & 22-09-14 & 23-09-14 & Componentes: & Los detalles de las \\
	Backlog & de usuarios&&& - Team & historias de usuario\\
	&&&& - Product Owner & \\
	&&&& - Scrum Master & \\
	&&&& - Cliente & \\
	\cline{2-4} \cline{6-6}
	& Gestion de prioridades& 24-09-14 & 24-09-14 && Los detalles  \\ 
	& (cliente) y peso(team)&&&&de la prioridades\\ 
	\cline{2-4} \cline{6-6}
	& Ordenamiento de & 25-09-14 & 25-09-14 && Los detalles \\
	& historias de usuario & &&& del ordenamiento. \\
	\cline{2-4} \cline{6-6}
	& Desgloze de& 26-09-14 & 26-09-14 && Los detalles \\
	& historias de usuario & &&& del desgloze. \\
	& en tareas &&&& \\
	\hline %Fin del Product Backlog
	Sprint I& Sprint Planning & 29-09-14 & 29-09-14 &  Componentes: & Los detalles del \\
	& Designacion de tareas &  &  & - Scrum Master & Sprint I \\
	& & & & - Team & \\
	& & & & - Product Owner & \\
	\cline{2-4} \cline{6-6}
	& Daily meeting & 30-09-14 & 10-10-14 & & Los detalles del  \\ 
	& Desarrollo de tareas &&&& Daily meeting\\
	& Burn down chart &&&& \\
	\hline %Fin Sprint I
	Sprint II& Sprint Planning & 13-10-14 & 13-09-14 &  Componentes: & Los detalles del \\
	& Designacion de tareas &  &  & - Scrum Master & Sprint I \\
	& & & & - Team & \\
	& & & & - Product Owner & \\
	\cline{2-4} \cline{6-6}
	& Daily meeting & 14-10-14 & 24-10-14 & & Los detalles del  \\ 
	& Desarrollo de tareas &&&& Daily meeting\\
	& Burn down chart &&&& \\
	\hline %Fin Sprint II
	Sprint III& Sprint Planning & 27-10-14 & 07-11-14 &  Componentes: & Los detalles del \\
	& Designacion de tareas &  &  & - Scrum Master & Sprint Planning \\
	& & & & - Team & \\
	& & & & - Product Owner & \\
	\cline{2-4} \cline{6-6}
	& Daily meeting & 30-09-14 & 24-10-14 & & Los detalles del  \\ 
	& Desarrollo de tareas &&&& Daily meeting\\
	& Burn down chart &&&& \\
	\hline %Fin Sprint III
	Sprint IV& Sprint Planning & 10-11-14 & 10-11-14 &  Componentes: & Los detalles del \\
	& Designacion de tareas &  &  & - Scrum Master & Sprint I \\
	& & & & - Team & \\
	& & & & - Product Owner & \\
	\cline{2-4} \cline{6-6}
	& Daily meeting & 11-11-14 & 21-11-14 & & Los detalles del  \\ 
	& Desarrollo de tareas &&&& Daily meeting\\
	& Burn down chart &&&& \\
	\hline %Fin Sprint IV
	
\end{tabular}\\
\newpage
\begin{tabular}{|c|p{4cm}|c|c|l|p{3cm}|}
	\hline
	Implatación & Transferencia de & 24-11-14 & 24-11-14 &  Componentes: & Los detalles del \\
	Puesta en & Tecnologia, Entrega de &  &  & - Scrum Master & Capacitacion \\
	marcha& Codigo Fuente & & & - Team & \\
	\cline{2-4} \cline{6-6}
	& Implantacion & 25-11-14 & 25-11-14 & & Los detalles del  \\ 
	& &&&& Daily meeting\\
	\cline{2-4} \cline{6-6}
	& Testeo en funcionamento  & 26-11-14 & 26-11-14 && Resultados \\
	\hline %Fin de la implantación
	Capacitación & Capacitación a  & 27-11-14 & 01-12-14 &  Componentes: & Los detalles de la \\
	& usuarios finales &  &  & - Scrum Master & Capacitación \\
	& & & & - Team & \\
	\hline %Fin Sprint IV
\end{tabular}

\chapter{PLAZO DE CONCLUSIÓN DE CONTRATO}
Work at home Soft se compromete a hacer la entrega final el dia 01 de diciembre de 2014.\\
Con posibilidad de extender el plazo de entrega hasta el dia la fecha 05 de diciembre de 2014 por motivos no previstos y dificultades en el proceso de desarrollo.

\chapter{PROPUESTA ECONÓMICA Y PLAN DE PAGOS}

\section{Propuesta económica}

Work at home se compromete a desarrollar el sistema de apoyo a la empresa TIS por un costo XXXBs. el cual sera remunerado segun como especifica el plan de pagos().
Para más detalle acerca del cálculo de los costos involucrados en este proyecto vease el anexo 6.2.

\section{Adelanto sobre la firma del contrato}

Work at home Soft solicita un adelanto del 15 \% del monto total a pagar en el momento de firma del contrato con el objetivo de cubrir los gastos iniciales del proyecto.

\section{Sobre las entregas}
Se entregara una versión incremental del sistema al final de cada iteración, cada versión será validada tanto por el cliente como por el equipo de desarrollo en una reunión conjunta, analizando en su conjunto el software entregado.
\section{Plan de pagos}
\begin{tabular}{|p{0.8cm}|p{1.5cm}|p{3cm}|p{0.95cm}|p{1cm}|p{3cm}|p{4cm}|}
	\hline
	 \multicolumn{7}{|c|}{Plan de Pagos} \\ \hline
	 \# de pago & Fecha de Pago & Item & Puntaje & Monto Bs. & Producto Entregable & Criterios de Aceptacion \\ \hline
	 1 & 2014-09-19 & Firma Contrato & 10 \% & 2.450 & Documento del contrato & Ambas partes en mutuo acuerdo con los terminos estipulados en el contrato. \\ \hline
	 2 & 2014-09-26 & Product Backlog & 10\% & 2.450 & Lista de todos los requerimientos. & El Product Backlog satisfaga las necesidades necesidades del cliente.Las historias de usuario deben ser no ambiguos y deben estar ordenados segun su retorno de la inversión(ROI).\\ \hline
	 3 & 2014-10-10 & Sprint I & 10\% & 2.450 & Historias de usuario del Sprint Backlog funcionando. & Las historias de usuario deben cumplir con sus respectivos criterios de acceptacion.El Sprint Backlog debe tener un 80\% de las historias de usuario terminadas.\\ \hline
	 4 & 2014-10-24 & Sprint II & 15 \% & 3.675 & Historias de usuario del Sprint Backlog funcionando. & Las historias de usuario deben cumplir con sus respectivos criterios de acceptacion.El Sprint Backlog debe tener un 80\% de las historias de usuario terminadas.\\ \hline
	 5 & 2014-11-07 & Sprint III & 15 \% & 3.675 & Historias de usuario del Sprint Backlog funcionando. & Las historias de usuario deben cumplir con sus respectivos criterios de acceptacion.El Sprint Backlog debe tener un 80\% de las historias de usuario terminadas.\\ \hline
	 6 & 2014-11-21 & Sprint IV & 20 \% & 4.900 & Historias de usuario del Sprint Backlog funcionando. & Las historias de usuario deben cumplir con sus respectivos criterios de acceptacion.El Sprint Backlog debe tener un 80\% de las historias de usuario terminadas.\\ \hline
	 
\end{tabular}
\newpage
\begin{tabular}{|p{0.8cm}|p{1.5cm}|p{3cm}|p{0.95cm}|p{1cm}|p{3cm}|p{4cm}|}
	\hline
	7 & 2014-11-26 & Capacitación & 10 \% & 2.450 & Manual de usuario.. & Los usuarios deben tener las nociones minimas de uso del sistema.\\ \hline
	8 & 2014-12-01 & Transferencia de tecnologia & 10 \% & 2.450 & Manual técnico. Manual de usuario. Manual de instalacion. Codigo fuente. & Sistema correctamente instalado. Manuales en orden y entendibles \\ \hline
	\multicolumn{3}{|c|}{\textbf{TOTAL PUNTAJE:}} & 100\% & 24.500& & \\ \hline
\end{tabular}
\chapter{DOCUMENTACION}
La documentación del manual de usuario esta de acuerdo a los roles especificos, \textbf{Adjuntado}.\\
TO-DO
\chapter{ANEXOS}
\section{Gestión de riesgos}
\section{Estimación De Costos Para El Sistema De Ayuda A La Empresa TIS}

\end{document}
