\documentclass[11pt,letterpaper]{report}
\usepackage[utf8]{inputenc}
\usepackage[spanish]{babel}
\usepackage{latexsym,amsmath,amssymb,amsthm}
\usepackage{graphicx}
\usepackage{pifont}
\usepackage[pdftex=true,colorlinks=true,plainpages=false]{hyperref}
\hypersetup{urlcolor=blue}
\hypersetup{linkcolor=black}
\hypersetup{citecolor=black}
\usepackage{lastpage}
\usepackage{url}
\usepackage{anysize}
\marginsize{3cm}{3cm}{1.2cm}{2.3cm}
\usepackage{fancyhdr}
\usepackage{anyfontsize}
\usepackage{tocbibind}
\usepackage{eso-pic}
\usepackage{mathptmx}
\usepackage{draftwatermark}
\usepackage{multirow}
\usepackage{pdfpages}
\usepackage{array}
\usepackage{booktabs}% http://ctan.org/pkg/booktabs
\newcommand{\tabitem}{~~\llap{\textbullet}~~}
\usepackage{multirow}
\usepackage{longtable}
\usepackage{lscape}
\usepackage{pdflscape}
%\usepackage[usenames,dvipsnames]{xcolor}
%\usepackage{lipsum}
%-------------------------
\setlength{\parindent}{0in}
\pagestyle{fancy}
\lhead{}
\chead{{\Huge \tt Work at Home Soft S.R.L.}\\~\\~ }%\\{\tt \huge ~}\\{\tt \Large -~}}
\rhead{}
\lfoot{}
\cfoot{{\small \tt Av. Maria del Carmen Rodriguez s/n, Zona Pacata Baja Cochabamba.\\ Teléfono: +591-70797024}\\{\Large \tt \url{http://github.com/workathome}}}
\rfoot{\thepage/\pageref{LastPage}}
\renewcommand{\headrule}{{\color{black}\hrule width\headwidth height\headrulewidth \vskip-\headrulewidth}}
\renewcommand{\footrule}{{\color{black}%
\vskip-\footruleskip\vskip-\footrulewidth
\hrule width\headwidth height\footrulewidth\vskip\footruleskip}}
\renewcommand{\headrulewidth}{3pt}
\renewcommand{\footrulewidth}{3pt}
\footskip = 32pt
\headheight = 56pt
\headsep = 8pt
%--------------------------


\newcommand\BackgroundPic{
\put(50,707){
\parbox[b][\paperheight]{\paperwidth}{%
%\vfill
%\centering
\includegraphics[scale=0.83]{img/logo.png}%
%\vfill
}}}

%---------------------------
\SetWatermarkAngle{0}
\SetWatermarkLightness{0.1}
%\SetWatermarkFontSize{1cm}
\SetWatermarkScale{3}
\SetWatermarkText{\includegraphics[scale=0.33]{img/marca_agua.png}}
%--------------------------------------

\begin{document}
\AddToShipoutPicture{\BackgroundPic}
%	\tableofcontents
~\\
~\\

\tableofcontents

\chapter{PROPUESTA DE SERVICIOS}
\section{Descripción del sistema a desarrollar}
La empresa TIS es una empresa que se dedica
al asesoramiento de grupos de desarrollo de software. Este trabajo consiste en acompañar el trabajo
de un grupo-empresa en el desarrollo de un software. La observación del trabajo de acompañamiento
permite a los asesores indicar los puntos débiles de un grupo empresa para que la calidad de su
proceso de desarrollo pueda mejorar.
La empresa TIS ha decidido desarrollar un sistema computacional para automatizar sus procesos.
Actualmente la forma en que suceden los procesos es la siguiente:
\begin{itemize}
	\item Publicación de la convocatoria pública.
	\item Publicación de lista de empresas inscritas en fundempresa de TIS.
	\item Publicación del pliego de especificaciones.
	\item Aclaraciones a la convocatoria y al pliego de especificaciones.
	\item Recepción de las propuestas e inscripción de las grupo empresas, cosiderando sus especificidades.
	\item Llevar una historia de las actividades en el desarrollo de la convocatoria.
	\item Permitir la emisión de  órdenes de cambio de acuerdo a convocatoria.
	\item Permitir la emisión de contratos.
	\item Permitir el registro de los avances semanales de las grupo empresas.
	\item Seguimiento de las grupo empresas, acorde a los que las GE definan como marco de trabajo. Una guia general puede ser tomando un modelo general de seguimiento de proyectos: Actividad, tarea, indicador,fecha de inicio, fecha fin, responsable, producto, observaciones.
\end{itemize}

\section{Objetivo General}

Desarrollar un sistema computacional que permita llevar adelante
los procesos administrativos y de gestión de proyectos de la empresa TIS.

\section{Objetivos especificos}

\section{Modalidad del proyecto}
Work at Home Soft presenta su propuesta de servicios en la modalidad de desarrollo de un producto de software, tal como exige el pliego de especificaciones PETIS-1707-2014.
\subsection{Proceso de desarrollo}
Para el desarrollo del sistema de Apoyo TIS se aplicara una metodologia de desarrollo agil é hibrida "Scrum + Kanban".\\
Las fases de desarrollo del proyecto son:\\
\begin{tabular}{ |c|p{4cm}|c|c|r|r| }
	\hline
	\multicolumn{6}{ |c| }{Planificación} \\
	\hline
	\textbf{Fases} & \textbf{Etapa} & \textbf{Fecha Inicio} & \textbf{Fecha Fin} & \textbf{Dias} & \textbf{Costo Bs.-}\\ \hline
	1 & Firma Contrato & 16-09-14 & 19-09-14 & 1& 2.450 \\ \hline
	2 & Product Backlog & 22-09-14 & 26-09-14 & 5&  2.450\\ \hline
	3 & Sprint I & 29-09-14 & 10-10-14 & 10&  2.450\\ \hline
	4 & Sprint II & 13-10-14 & 24-10-14 & 10&  3.675\\ \hline
	5 & Sprint III & 27-10-14 & 07-11-14 & 10&  3.675\\ \hline
	6 & Sprint IV & 10-11-14 & 21-11-14 & 10&  4.900\\ \hline
	7 & Transferencía de tecnologia, implantacion y puesta en marcha & 24-11-14 & 26-11-14 & 3& 2.450 \\ \hline
	8 & Capacitación & 27-11-14 & 01-12-14 & 3& 2.450 \\ \hline
	\multicolumn{4}{|c|}{Total dias hábiles laborales} & 52 & 24.500 \\ \hline
\end{tabular}
\section{Generalidades del sistema}

\subsection{Gestión de Base de Datos}
TO-DO
\subsection{Gestión de Información}
TO-DO

\section{Software para el desarrollo}
\subsection{Licencias de software}
Work at Home Soft en cumplimiento con el pliego de especificaciones asegura que la tecnologia y herramientas a utilizar  en el desarrollo del sistema tienen licencia libre, a continuacion se detalla:\\
\\
\begin{tabular}{ lccc }
	\multicolumn{4}{c}{Tecnologia de desarrollo} \\
	\hline
	\textbf{Descripción} & \textbf{Tecnologia} & \textbf{Licencia} & \textbf{Referecia} \\ \hline
	Lenguaje de Programación & PHP 5 & PHP License & \url{http://php.net/copyright.php}\\
	Framework Backend & Laravel & MIT License & \url{http://opensource.org/licenses/MIT} \\  
	Gestor de Base de Datos & Mariadb & GPL License & \url{http://www.gnu.org/copyleft/gpl.html} \\
	Diseño de Base de Datos & Mysql-WorkBench & GPL License & \url{http://www.gnu.org/copyleft/gpl.html}\\
	Control de Versiones & Git & GPL License & \url{http://www.gnu.org/copyleft/gpl.html}\\
	Entorno de desarrollo & Vim & GPL-compatible & \url{http://www.gnu.org/copyleft/gpl.html}\\
	
	
\end{tabular}
\\
\\Ademas de que el sistema desarrollado en la primera fase ya cumple con este requerimiento.
\section{Metodología del desarrollo}
Se usara una metodologia de desarrollo hibrida, una mezcla de Scrum y Kanban.\\
A continuacion se detalla la adopción de esta.
\subsubsection{Scrum:} 
Se usara lo siguiente:
\begin{itemize}
	\item Reuniones establecidas en scrum(Daily Meeting, Sprint Planning Meeting, Sprint Review Meeting, Sprint Retrospective).
	\item Roles de scrum(Product Owner, ScrumMaster, Team).
	\item Documentos de scrum(Product backlog, Sprint backlog, Burn down chart).
\end{itemize}
\subsubsection{Kanban:} 
\begin{itemize}
	\item Tablero Kanban.\\
	Un tablero Kanban, se divide en columnas las cuales representan un proceso de trabajo. En nuestro caso las columnas se dividen en Cola de Espera, Analisis, Desarrollo, Pruebas, Finalizado.
	\item Uso correcto de WIP( Work in Progress) Trabajo en progreso.\\
	Consisten en acordar anticipadamente, la cantidad de ítems que pueden abordarse por cada proceso (es decir, por columnas del tablero).\\
	El principal objetivo de establecer estos límites, es el de detectar cuellos de botella. 
	\item Optimizacion del Flujo de trabajo.\\
	El objetivo una la producción estable, continua y previsible. Midiendo el tiempo que el ciclo completo de ejecución del proyecto demanda, se obtiene el CycleTime(tiempo de ciclo).
	\begin{equation}
	Throughput(rendimiento) = CycleTime/WIP
	\end{equation}
	Con estos valores, la optimización del flujo de trabajo consistirá en la búsqueda de:
	\begin{enumerate}
		\item Minimizar el CycleTime(tiempo de ciclo).
		\item Maximizar el Throughput(redimeinto).
		\item Lograr una variabilidad mínima entre CycleTime y Throughput
	\end{enumerate}
\end{itemize}

\chapter{PLANIFICACIÓN}
\section{Planificacion global del proyecto}
A continuacion se presenta la planificación global de actividades acorde con nuestro proceso de desarrollo.
\begin{center}
\begin{tabular}{|c|p{4.5cm}|c|p{6cm}|}
	\hline
	\multicolumn{4}{|c|}{Planificación global} \\
	\hline
	\ Numero & Actividad & Duracion(Dias) & Hitos \\ \hline
	A1 & Firma del contrato & 1 & Contrato firmado. \\ \hline
	A2 & Product Backlog & 5 & Documento del Product Backlog. \\ \hline
	A3 & Sprint I & 10 & Primera versión del sistema. \\ \hline
	A4 & Sprint II & 10 & Segunda versión del sistema. \\ \hline
	A5 & Sprint III & 10 & Tercera versión del sistema. \\ \hline
	A6 & Sprint IV & 10 & Versión final del sistema. \\ \hline
	A7 & Transferencía de tecnologia, implantacion y puesta en marcha. & 3 & Implantacion de sistema, Codigo fuente, manual técnico  de usuario y de instalación. \\  \hline
	A8 & Capacitación & 3 &  Capacitacion a los usuarios finales.\\ \hline
\end{tabular}
\end{center}
\section{Planificacion especifica del proyecto}
\begin{tabular}{|c|p{4cm}|c|c|l|p{3cm}|}
	\hline
	\multicolumn{6}{|c|}{Planificacion especifica del proyecto} \\ \hline
	Actividad & Plan & Fecha Inicio & Fecha Fin & Responsables & Detalle \\ \hline
	Product & Gestion de Historias  & 22-09-14 & 23-09-14 & Componentes: & Los detalles de las \\
	Backlog & de usuarios&&& - Team & historias de usuario\\
	&&&& - Product Owner & \\
	&&&& - Scrum Master & \\
	&&&& - Cliente & \\
	\cline{2-4} \cline{6-6}
	& Gestion de prioridades& 24-09-14 & 24-09-14 && Los detalles  \\ 
	& (cliente) y peso(team)&&&&de la prioridades\\ 
	\cline{2-4} \cline{6-6}
	& Ordenamiento de & 25-09-14 & 25-09-14 && Los detalles \\
	& historias de usuario & &&& del ordenamiento. \\
	\cline{2-4} \cline{6-6}
	& Desgloze de& 26-09-14 & 26-09-14 && Los detalles \\
	& historias de usuario & &&& del desgloze. \\
	& en tareas &&&& \\
	\hline %Fin del Product Backlog
	Sprint I& Sprint Planning & 29-09-14 & 29-09-14 &  Componentes: & Los detalles del \\
	& Designacion de tareas &  &  & - Scrum Master & Sprint I \\
	& & & & - Team & \\
	& & & & - Product Owner & \\
	\cline{2-4} \cline{6-6}
	& Daily meeting & 30-09-14 & 10-10-14 & & Los detalles del  \\ 
	& Desarrollo de tareas &&&& Daily meeting\\
	& Burn down chart &&&& \\
	\hline %Fin Sprint I
	Sprint II& Sprint Planning & 13-10-14 & 13-09-14 &  Componentes: & Los detalles del \\
	& Designacion de tareas &  &  & - Scrum Master & Sprint I \\
	& & & & - Team & \\
	& & & & - Product Owner & \\
	\cline{2-4} \cline{6-6}
	& Daily meeting & 14-10-14 & 24-10-14 & & Los detalles del  \\ 
	& Desarrollo de tareas &&&& Daily meeting\\
	& Burn down chart &&&& \\
	\hline %Fin Sprint II
	Sprint III& Sprint Planning & 27-10-14 & 07-11-14 &  Componentes: & Los detalles del \\
	& Designacion de tareas &  &  & - Scrum Master & Sprint Planning \\
	& & & & - Team & \\
	& & & & - Product Owner & \\
	\cline{2-4} \cline{6-6}
	& Daily meeting & 30-09-14 & 24-10-14 & & Los detalles del  \\ 
	& Desarrollo de tareas &&&& Daily meeting\\
	& Burn down chart &&&& \\
	\hline %Fin Sprint III
	Sprint IV& Sprint Planning & 10-11-14 & 10-11-14 &  Componentes: & Los detalles del \\
	& Designacion de tareas &  &  & - Scrum Master & Sprint I \\
	& & & & - Team & \\
	& & & & - Product Owner & \\
	\cline{2-4} \cline{6-6}
	& Daily meeting & 11-11-14 & 21-11-14 & & Los detalles del  \\ 
	& Desarrollo de tareas &&&& Daily meeting\\
	& Burn down chart &&&& \\
	\hline %Fin Sprint IV
	
\end{tabular}\\
\newpage
\begin{tabular}{|c|p{4cm}|c|c|l|p{3cm}|}
	\hline
	Implatación & Transferencia de & 24-11-14 & 24-11-14 &  Componentes: & Los detalles del \\
	Puesta en & Tecnologia, Entrega de &  &  & - Scrum Master & Capacitacion \\
	marcha& Codigo Fuente & & & - Team & \\
	\cline{2-4} \cline{6-6}
	& Implantacion & 25-11-14 & 25-11-14 & & Los detalles del  \\ 
	& &&&& Daily meeting\\
	\cline{2-4} \cline{6-6}
	& Testeo en funcionamento  & 26-11-14 & 26-11-14 && Resultados \\
	\hline %Fin de la implantación
	Capacitación & Capacitación a  & 27-11-14 & 01-12-14 &  Componentes: & Los detalles de la \\
	& usuarios finales &  &  & - Scrum Master & Capacitación \\
	& & & & - Team & \\
	\hline %Fin Sprint IV
\end{tabular}

\chapter{PLAZO DE CONCLUSIÓN DE CONTRATO}
Work at home Soft se compromete a hacer la entrega final el dia 01 de diciembre de 2014.\\
Con posibilidad de extender el plazo de entrega hasta el dia la fecha 05 de diciembre de 2014 por motivos no previstos y dificultades en el proceso de desarrollo.

\chapter{PROPUESTA ECONÓMICA Y PLAN DE PAGOS}

\section{Propuesta económica}

Work at home se compromete a desarrollar el sistema de apoyo a la empresa TIS por un costo XXXBs. el cual sera remunerado segun como especifica el plan de pagos().
Para más detalle acerca del cálculo de los costos involucrados en este proyecto vease el anexo 6.2.

\section{Adelanto sobre la firma del contrato}

Work at home Soft solicita un adelanto del 15 \% del monto total a pagar en el momento de firma del contrato con el objetivo de cubrir los gastos iniciales del proyecto.

\section{Sobre las entregas}
Se entregara una versión incremental del sistema al final de cada iteración, cada versión será validada tanto por el cliente como por el equipo de desarrollo en una reunión conjunta, analizando en su conjunto el software entregado.
\section{Plan de pagos}
\begin{tabular}{|p{0.8cm}|p{1.5cm}|p{3cm}|p{0.95cm}|p{1cm}|p{3cm}|p{4cm}|}
	\hline
	 \multicolumn{7}{|c|}{Plan de Pagos} \\ \hline
	 \# de pago & Fecha de Pago & Item & Puntaje & Monto Bs. & Producto Entregable & Criterios de Aceptacion \\ \hline
	 1 & 2014-09-19 & Firma Contrato & 10 \% & 2.450 & Documento del contrato & Ambas partes en mutuo acuerdo con los terminos estipulados en el contrato. \\ \hline
	 2 & 2014-09-26 & Product Backlog & 10\% & 2.450 & Lista de todos los requerimientos. & El Product Backlog satisfaga las necesidades necesidades del cliente.Las historias de usuario deben ser no ambiguos y deben estar ordenados segun su retorno de la inversión(ROI).\\ \hline
	 3 & 2014-10-10 & Sprint I & 10\% & 2.450 & Historias de usuario del Sprint Backlog funcionando. & Las historias de usuario deben cumplir con sus respectivos criterios de acceptacion.El Sprint Backlog debe tener un 80\% de las historias de usuario terminadas.\\ \hline
	 4 & 2014-10-24 & Sprint II & 15 \% & 3.675 & Historias de usuario del Sprint Backlog funcionando. & Las historias de usuario deben cumplir con sus respectivos criterios de acceptacion.El Sprint Backlog debe tener un 80\% de las historias de usuario terminadas.\\ \hline
	 5 & 2014-11-07 & Sprint III & 15 \% & 3.675 & Historias de usuario del Sprint Backlog funcionando. & Las historias de usuario deben cumplir con sus respectivos criterios de acceptacion.El Sprint Backlog debe tener un 80\% de las historias de usuario terminadas.\\ \hline
	 6 & 2014-11-21 & Sprint IV & 20 \% & 4.900 & Historias de usuario del Sprint Backlog funcionando. & Las historias de usuario deben cumplir con sus respectivos criterios de acceptacion.El Sprint Backlog debe tener un 80\% de las historias de usuario terminadas.\\ \hline
	 
\end{tabular}
\newpage
\begin{tabular}{|p{0.8cm}|p{1.5cm}|p{3cm}|p{0.95cm}|p{1cm}|p{3cm}|p{4cm}|}
	\hline
	7 & 2014-11-26 & Capacitación & 10 \% & 2.450 & Manual de usuario.. & Los usuarios deben tener las nociones minimas de uso del sistema.\\ \hline
	8 & 2014-12-01 & Transferencia de tecnologia & 10 \% & 2.450 & Manual técnico. Manual de usuario. Manual de instalacion. Codigo fuente. & Sistema correctamente instalado. Manuales en orden y entendibles \\ \hline
	\multicolumn{3}{|c|}{\textbf{TOTAL PUNTAJE:}} & 100\% & 24.500& & \\ \hline
\end{tabular}
\chapter{DOCUMENTACION}
La documentación del manual de usuario esta de acuerdo a los roles especificos, \textbf{Adjuntado}.\\
TO-DO
\chapter{ANEXOS}
\section{Gestión de riesgos}
\subsection{Análisis de Probabilidad e impacto}
\begin{tabular}{|c|l|c|c|}
	\hline
	\multicolumn{4}{|c|}{Gestión de riesgos}\\ \hline
	Código & Riesgos Posibles & Probabilidad & Impacto \\ 
	& & de ocurrencia & \\ \hline
	R1 & Mala estimación de tiempos  & 50\% & medio \\ \hline
	R2 & Ausencia de un integrante del equipo & 1\% & alto \\ \hline
	R3 & Desgaste de energía del equipo de desarrollo & 10\% & medio  \\ \hline
	R4 & No llevar a cabo regularmente revisiones técnicas  & 10\% & media \\ 
	& formales de las especificaciones de requisitos, diseño y código & & \\ \hline
	R5 & mala documentación de los resultados de las revisiones técnicas,  & 10\% & medio \\ 
	& incluyendo errores encontrados y recursos empleados. && \\ \hline
	R6 & Falta de experiencia en el uso de la plataforma & 6\% & medio \\ \hline
	R7 & Falta de especificaciones de las funciones en el código & 60\% & bajo \\ \hline
	R8 & Insuficiencia de recursos económicas & 20\% & alto \\ \hline
	R9 & Fallas técnicas de las computadoras & 10\% & alto \\ \hline
	R10 & Cambio o aumento de requerimientos por parte del cliente & 60\% & medio \\ \hline
	R11 & Interfaz rechazada por el usuario & 60\% & alto \\ \hline
	R12 & Software no cumple con algún requerimiento & 30\% & alto \\ \hline
	R13 & Poca adaptabilidad al sistema por parte de los usuarios & 10\% & alto  \\ \hline
	R14 & Los documentos (Manual de Usuario) puede no  & 15\% & medio \\ 
	& ser entendible para el usuario. && \\ \hline
	R15 & No se tiene el apoyo por parte del cliente & 5\% & alto \\ \hline
	R16 & El cliente no tenga idea de lo que realmente quiere/necesita & 50\% & alto \\ \hline
	R17 & Fallas en el servidor & 50\% & alto \\ \hline
	R18 & La tecnologia no alcance las espectativas del cliente & 25\% & alto \\ \hline
	R19 & Personal clave enfermo o no disponible en momentos criticos & 30\% & medio \\ \hline
	R20 & No disponibilidad de Hardware & 35\% & medio \\ \hline
	R21 & Cambio de Tecnologia & 40\% & medio \\ \hline
	R22 & Personal inexperto & 5\% & medio \\ \hline
\end{tabular}

\subsection{Análisis de las Causas}

\begin{longtable}{|l|l|}
	\hline
	\multicolumn{1}{|c|}{\textbf{RIESGO}} & \multicolumn{1}{c|}{\textbf{CAUSA}} \\ \hline
  	\endhead
	R1 & \begin{minipage}{5in}
	    \vskip 1pt
	    \begin{enumerate}
	   \item Mala planificación del proyecto
	   \item Imprevistos(paro de transporte, etc)
	   \end{enumerate}
	   \vskip 1pt
	 \end{minipage}\\\hline 
	R2 & \begin{minipage}{5in}
		    \vskip 1pt
		    \begin{enumerate}
		   \item Deterioro de salud
		   \item Problemas familiares
		   \end{enumerate}
		   \vskip 1pt
		 \end{minipage}\\\hline
	R8 & \begin{minipage}{5in}
			    \vskip 1pt
			    \begin{enumerate}
			   \item Mala estimación de presupuesto
			   \item Mala administración de presupuesto
			   \end{enumerate}
			   \vskip 1pt
			 \end{minipage}\\\hline
	R9 & \begin{minipage}{5in}
			    \vskip 1pt
			    \begin{enumerate}
			   \item Falta de mantenimiento de los equipos
			   \item Falla de fábrica
			   \end{enumerate}
			   \vskip 1pt
			 \end{minipage}\\\hline
	R10 & \begin{minipage}{5in}
			    \vskip 1pt
			    \begin{enumerate}
			   \item Falta de seriedad del cliente
			   \item Rechazo a requerimientos identificados
			   \item Ambigüedad en ls requerimientos
			   \end{enumerate}
			   \vskip 1pt
			 \end{minipage}\\\hline
	R11 & \begin{minipage}{5in}
				    \vskip 1pt
				    \begin{enumerate}
				   \item Poca comunicación con el cliente
				   \item Falta de experiencia en el diseño de interfaz
				   \end{enumerate}
				   \vskip 1pt
				 \end{minipage}\\\hline 
	R12 & \begin{minipage}{5in}
				    \vskip 1pt
				    \begin{enumerate}
				   \item Falta de tiempo en el desarrollo
				   \item Falta de presupuesto
				   \item Ambigüedad en el requerimiento
				   \end{enumerate}
				   \vskip 1pt
				 \end{minipage}\\\hline
	R15 & \begin{minipage}{5in}
				    \vskip 1pt
				    \begin{enumerate}
				   \item Falta de disponibilidad de tiempo
				   \item Falta de comunicación
				   \item Rencilla entre cliente - equipo 
				   \end{enumerate}
				   \vskip 1pt
				 \end{minipage}\\\hline
	R16 & \begin{minipage}{5in}
				    \vskip 1pt
				    \begin{enumerate}
				   \item El cliente no conoce el alcance de la tecnología
				   \item No sabe expresar sus ideas
				   \item No conoce la necesidad de sus empleados.
				   \end{enumerate}
				   \vskip 1pt
				 \end{minipage}\\\hline
	R17 & \begin{minipage}{5in}
				    \vskip 1pt
				    \begin{enumerate}
				   \item Falta de mantenimiento preventivo al servidor
				   \item Mala administración del servidor
				   \item Ambiente inadecuado para el servidor
				   \end{enumerate}
				   \vskip 1pt
				 \end{minipage}\\\hline
	R18 & \begin{minipage}{5in}
				    \vskip 1pt
				    \begin{enumerate}
				   \item Servidor de producción limitado
				   \item Falta de conocimientos sobre nuevas tecnologías cliente/equipo
				   \item Presupuesto insuficiente para la compra de equipos
				   \end{enumerate}
				   \vskip 1pt
				 \end{minipage}\\\hline
\end{longtable}


\subsection{Plan de Contingencias}

\begin{longtable}{|l|p{1in}| p{1in} |p{0.7in}|}
	\hline
	\multicolumn{1}{|c|}{\bf ACTIVIDAD} & \multicolumn{1}{c|}{\bf TIEMPO} & \multicolumn{1}{c|}{\bf RECURSOS} & \multicolumn{1}{c|}{\bf RESPONSABLE} \\ \hline
  	\endhead
  	 \begin{minipage}{2.5in}
	    \vskip 6pt
		\underline{R1 Tiempo estimado demasiado pequeño}
	    \begin{itemize}
	   \item[{\bf A1}] Investigar la compra de componentes (SW) externos.
	   \item[{\bf A2}] Descomposición en tareas más
	   \item[{\bf R3}] Reusar SW
	   \end{itemize}
	   \vskip 1pt
	 \end{minipage}
	 & Semanalmente & Documento de planificación de proyectos & jefe del proyecto \\\hline 
	 %-----------------
	 \begin{minipage}{2.5in}
	 	    \vskip 6pt
	 	    \underline{R2 Ausencia de un integrante del equipo}
	 	    \begin{itemize}
	 	   \item[{\bf A1}] Motivar al equipo constantemente
	 	   \item[{\bf A2}] Motivar al equipo constantemente
	 	   \end{itemize}
	 	   \vskip 1pt
	 	 \end{minipage}		
	 	 & Semanalmente & Recursos económicos & Jefe del proyecto y el equipo \\\hline
	 %-----------------
	 \begin{minipage}{2.5in}
	 	\vskip 6pt
	    \underline{R8 Insuficiencia de recursos económicas}
	    \begin{itemize}
		    \item[{\bf A1}] Realizar cuidadosamente una estimación de recursos
	   		\item[{\bf A2}] Contar con reservas económicas
	   		\item[{\bf A3}] Hacer respetar las fechas de pago previa entrega y validación de producto
	    \end{itemize}
	   	\vskip 1pt
	 \end{minipage}	
	& Antes de la planificación & Departamento de Administración	& Jefe de Proyecto\\\hline

	%-----------------
	 \begin{minipage}{2.5in}
		\vskip 6pt
	  	\underline{ R9 Fallas técnicas de las computadoras }
	  \begin{itemize}
	   		\item[{\bf A1}] Realizar mantenimiento preventivo
	   		\item[{\bf A2}] Adquirir equipos garantizados y con buen soporte técnico 
	  \end{itemize}
		   	\vskip 1pt
	 \end{minipage}	
		& Quincenalmente & Herramientas e insumos adecuados para el mantenimiento de las computadoras & Todo el equipo\\\hline
	%-----------------
	 \begin{minipage}{2.5in}
		\vskip 6pt
	  	\underline{ R10 ~ Cambio o aumento de }
	  	\underline{requerimientos por parte del cliente}
	  \begin{itemize}
	   		\item[{\bf A1}] Utilizar una metodología que se adapte a los cambios
	  \end{itemize}
		   	\vskip 1pt
	 \end{minipage}	
		& Antes de la planificación & Cliente	& Equipo y Jefe de Proyecto\\\hline

	%-----------------
	 \begin{minipage}{2.5in}
		\vskip 6pt
	  	\underline{R11 Interfaz rechazada por el usuario}
	  \begin{itemize}
	   		\item[{\bf A1}] La interfaz del usuario debe contemplar la visión de la empresa (logotipo, color, tipo de fuente)
	   		\item[{\bf A2}] Capacitación del equipo sobre la creación de interfaces
	  \end{itemize}
		   	\vskip 1pt
	 \end{minipage}	
		& Durante el diseño de las interfaces &	Tiempo del cliente Herramientas de prototipado para la interfaz & Equipo y el cliente\\\hline


	%-----------------
	 \begin{minipage}{2.5in}
		\vskip 6pt
	  	\underline{R12 ~ Software no cumple con algún}
	  	\underline{requerimiento}
	  \begin{itemize}
	   		\item[{\bf A1}] Realizar cuidadosamente la planificación
	   		\item[{\bf A2}] Revisión minuciosa de los requerimientos
	  \end{itemize}
		   	\vskip 1pt
	 \end{minipage}	
		& Durante la Planificación & Herramientas de seguimiento del proyecto & Equipo\\\hline

	%-----------------
	 \begin{minipage}{2.5in}
		\vskip 6pt
	  	\underline{R15. No se tiene el apoyo por parte del }
	  	\underline{cliente}
	  \begin{itemize}
	   		\item[{\bf A1}] Planificar y aprovechar las reuniones del cliente
	   		\item[{\bf A2}] Comprometer al cliente
	  \end{itemize}
		   	\vskip 1pt
	 \end{minipage}	
		& Durante el desarrollo	& Cliente	& Jefe de  Proyecto y equipo\\\hline


	%-----------------
	 \begin{minipage}{2.5in}
		\vskip 6pt
	  	\underline{R16. El cliente no tenga idea de}
	  	\underline{lo que realmente quiere/necesita}
	  \begin{itemize}
	   		\item[{\bf A1}] Observar como el usuario realiza su trabajo
	   		\item[{\bf A2}] Mostrar al usuario sistemas similares al que desea
	  \end{itemize}
		   	\vskip 1pt
	 \end{minipage}	
		& Al inicio de la etapa de definición del proyecto & Cuestionarios & El equipo\\\hline	
		
		%-----------------
	 \begin{minipage}{2.5in}
		\vskip 6pt
	  	\underline{R17. Fallas en el servidor}
	 	\begin{itemize}
	   		\item[{\bf A1}] Verificar que el servidor funcione correctamente
	   		\item[{\bf A2}] Verificar que el servidor cuente con los paquetes adecuados
	 	\end{itemize}
		\vskip 1pt
	 \end{minipage}	
			& Antes de la implantación del Software	& Hardware y Software Acceso a los servidores & Jefe del Proyecto, Cliente, Equipo\\\hline	

		%-----------------
	 \begin{minipage}{2.5in}
		\vskip 6pt
	  	\underline{R18 La tecnología no alcance las}
	  	\underline{ espectativas del cliente }
	 	\begin{itemize}
	   		\item[{\bf A1}] Verificar la capacidad y limitaciones del servidor 
	   		\item[{\bf A2}] Actualización y capacitación constante del equipo
	   		\item[{\bf A3}] Mostrar al cliente las nuevas tecnologías y ver si este esta dispuesto a pagar el costo de estas
	 	\end{itemize}
		\vskip 1pt
	 \end{minipage}	
			& Antes de la planificación al inicio el proyecto & &		\\\hline			


\end{longtable}
\section{Estimación De Costos Para El Sistema De Ayuda A La Empresa TIS}

\end{document}
